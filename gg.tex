\documentclass[bachelor, och, diploma]{SCWorks}
% параметр - тип обучения - одно из значений:
%    spec     - специальность
%    bachelor - бакалавриат (по умолчанию)
%    master   - магистратура
% параметр - форма обучения - одно из значений:
%    och   - очное (по умолчанию)
%    zaoch - заочное
% параметр - тип работы - одно из значений:
%    referat    - реферат
%    coursework - курсовая работа (по умолчанию)
%    diploma    - дипломная работа
%    pract      - отчет по практике
%    pract      - отчет о научно-исследовательской работе
%    autoref    - автореферат выпускной работы
%    assignment - задание на выпускную квалификационную работу
%    review     - отзыв руководителя
%    critique   - рецензия на выпускную работу
% параметр - включение шрифта
%    times    - включение шрифта Times New Roman (если установлен)
%               по умолчанию выключен
\usepackage[T2A]{fontenc}
\usepackage[utf8]{inputenc}
\usepackage{graphicx}
\usepackage{underscore}
\usepackage{listings}
\usepackage{xcolor}
\lstset { %
    language=C++,
    %backgroundcolor=\color{black!5}, % set backgroundcolor
    %basicstyle=\footnotesize,% basic font setting
    basicstyle=\small
}

\usepackage[sort,compress]{cite}
\usepackage{amsmath}
\usepackage{amssymb}
\usepackage{amsthm}
\usepackage{fancyvrb}
\usepackage{longtable}
\usepackage{array}
\usepackage[english,russian]{babel}
\usepackage{minted}
% Используется автором репозитория
%\usemintedstyle{xcode}
% Этот пакет включает в себя аналогичный Times New Roman шрифт.
% Необходим для успешной компиляции для UNIX-систем ввиду отсутствия TNR в нем.
% Можно использовать и для Windows.
\usepackage{tempora}

\usepackage{caption}

\usepackage[colorlinks=false]{hyperref}

\usepackage{verbatim}

\newcommand{\eqdef}{\stackrel {\rm def}{=}}

\newtheorem{lem}{Лемма}

% % При использовании biblatex вместо bibtex
% \usepackage[style=gost-numeric]{biblatex}
% \addbibresource{thesis.bib}

\begin{document}

% Кафедра (в родительном падеже)
\chair{дискретной математики и информационных технологий}

% Тема работы
\title{Анализ звуковых волн с использованием рядов Фурье и Python}

% Курс
\course{4}

% Группа
\group{421}

% Факультет (в родительном падеже) (по умолчанию "факультета КНиИТ")
%\department{факультета КНиИТ}

% Специальность/направление код - наименование
%\napravlenie{02.03.02 "--- Фундаментальная информатика и информационные 
% технологии}
% \napravlenie{02.03.03 "--- Математическое обеспечение и администрирование 
% информационных систем}

\napravlenie{09.03.01 "--- дискретной математики и информационных технологий}
%\napravlenie{09.03.04 "--- Программная инженерия}
%\napravlenie{10.05.01 "--- Компьютерная безопасность}

% Для студентки. Для работы студента следующая команда не нужна.
%\studenttitle{Студентки}

% Фамилия, имя, отчество в родительном падеже
\author{Космынина Константина Алексеевича}

% Заведующий кафедрой
\chtitle{доцент, к.\,ф.-м.\,н.} % степень, звание
\chname{Л.\,Б.\,Тяпаев}

%Научный руководитель (для реферата преподаватель проверяющий работу)
\satitle{профессор, д.\,ф.-м.\,н.} %должность, степень, звание
\saname{В.\,А.\,Молчанов}

% Руководитель практики от организации (только для практики,
% для остальных типов работ не используется)
% \patitle{Заместитель генерального директора}
% \patitlecompany{ООО <<Базальт СПО>>}
% \paname{Е.\,А.\,Синельников}

% Семестр (только для практики, для остальных
% типов работ не используется)
% \scourse{3}
% \term{6}

% Наименование практики (только для практики, для остальных
% типов работ не используется)
% \practtype{технологическая (проектно-технологическая) практика
% (производственная практика)}

% Продолжительность практики (количество недель) (только для практики,
% для остальных типов работ не используется)
% \duration{4}

% Даты начала и окончания практики (только для практики, для остальных
% типов работ не используется)
% \practStart{22.06.2024}
% \practFinish{19.07.2024}

% Год выполнения отчета
\date{2025}

\maketitle

% Включение нумерации рисунков, формул и таблиц по разделам
% (по умолчанию - нумерация сквозная)
% (допускается оба вида нумерации)
%\secNumbering

\tableofcontents

% Раздел "Обозначения и сокращения". Может отсутствовать в работе
% \abbreviations
% \begin{description}
%     \item ... "--- ...
%     \item ... "--- ...
% \end{description}

% Раздел "Определения". Может отсутствовать в работе
% \definitions

% Раздел "Определения, обозначения и сокращения". Может отсутствовать в работе.
% Если присутствует, то заменяет собой разделы "Обозначения и сокращения" и 
% "Определения"
\defabbr

% Раздел "Введение"

\noindent \textbf{Звуковая волна} --- периодическое изменение давления в среде (например, воздухе), воспринимаемое человеческим ухом как звук.\\[0.5em]
\textbf{Частота} --- число колебаний (повторений периода) звуковой волны в единицу времени, измеряется в герцах (Гц). Определяет высоту звука.\\[0.5em]
\textbf{Амплитуда} --- максимальное отклонение колебания звуковой волны от среднего значения. Определяет громкость звука.\\[0.5em]
\textbf{Гармоника} --- синусоидальная составляющая звуковой волны, частота которой является целым кратным основной частоте.\\[0.5em]
\textbf{Ряд Фурье} --- разложение периодической функции во временной области в сумму синусоидальных функций (гармоник) с различными амплитудами и фазами.\\[0.5em]
\textbf{Дискретное преобразование Фурье (DFT)} --- вычисление коэффициентов рядов Фурье по дискретным отсчётам сигнала. В контексте настоящей работы DFT используется для анализа цифровых звуковых данных.\\[0.5em]
\textbf{Быстрое преобразование Фурье (FFT)} --- алгоритм быстрого вычисления дискретного преобразования Фурье, разработанный Кули и Тьюки (1965 г.), существенно сокращающий время расчёта гармонического состава сигналов.\\[0.5em]
\textbf{Python} --- высокоуровневый язык программирования, используемый в работе для обработки сигналов.\\[0.5em]
\textbf{PyGame} --- библиотека Python для разработки мультимедийных приложений, в том числе позволяет работать со звуковыми данными (воспроизведение звука из массивов чисел).\\[0.5em]
\textbf{NumPy} --- библиотека Python для научных вычислений, предоставляющая удобный функционал для работы с массивами (используется для хранения дискретных сигналов).\\[0.5em]
\textbf{Дискретизация (самплирование)} --- процесс преобразования непрерывного сигнала (звуковой волны) в последовательность дискретных отсчётов (цифровой сигнал) с заданной частотой дискретизации.\\[0.5em]
\textbf{Частота дискретизации} --- число отсчётов звукового сигнала в секунду при его дискретизации, измеряется в Гц. В работе используется стандартная аудио-частота 44100 Гц.\\[0.5em]
\textbf{Битовая глубина} --- разрядность представления амплитуды цифрового звука (16-бит --- целые числа от -32768 до 32767). Определяет динамический диапазон (точность представления амплитуды).


\intro

Актуальность темы обусловлена широким применением анализа звуковых сигналов в науке и технике. Звуковые волны являются одним из важнейших видов сигналов, и их исследование необходимо для понимания музыкальных тонов, речи, а также для разработки современных технологий обработки аудио (сжатие, фильтрация, синтез звука и т.п.). Ряды Фурье представляют мощный математический аппарат для анализа периодических сигналов, позволяя разложить сложную звуковую волну на простые гармонические составляющие. Такой подход лежит в основе спектрального анализа звука, который широко применяется в акустике и цифровой обработке сигналов \cite{fletcher}, \cite{rossing}. Сочетание методов гармонического анализа с возможностями языка Python (и его библиотек) делает задачу исследования звуковых волн доступной и наглядной. Использование языка Python позволяет не только проводить математические расчёты, но и сразу воспроизводить звуковые сигналы, что обеспечивает наглядную проверку теоретических выводов на практике.

\textbf{Цель работы} --- провести анализ звуковых волн с использованием разложения в ряд Фурье и методов программной обработки сигналов на языке Python. Для достижения цели необходимо решить следующие \textbf{задачи}:
\begin{itemize}
    \item Изучить теоретические основы представления звуковых сигналов и рядов Фурье: понятие звуковой волны, гармоники, спектра, математический аппарат рядов Фурье.
    \item Проанализировать современные подходы к анализу звука и существующие решения, представленные в литературе и источниках (как отечественных, так и зарубежных).
    \item Разработать методологию экспериментального анализа звуковых волн с помощью Python: выбрать инструменты (библиотеки) для генерации и воспроизведения звука, для вычисления коэффициентов Фурье.
    \item Реализовать программные примеры: генерация простейших звуковых сигналов (моночастотных синусоид), их воспроизведение; объединение нескольких сигналов (формирование аккордов) и исследование результирующей волны; разложение сложной звуковой волны на гармоники (вычисление ряда Фурье) с идентификацией составляющих частот.
    \item Разработать и реализовать интерактивное веб-приложение 'fourier-wave-analyzer' для демонстрации анализа и синтеза звуковых волн на основе рядов Фурье.
    % \item Проанализировать результаты вычислительного эксперимента: сравнить исходные сигналы и восстановленные по ряду Фурье, оценить точность разложения, продемонстрировать практические применения (например, определение нотного состава звука).
\end{itemize}

\textbf{Объект исследования} --- звуковые волны как физическое и математическое явление, рассматриваемые в непрерывной и дискретной форме. \textbf{Предмет исследования} --- методы анализа звуковых волн на основе разложения в ряд Фурье и алгоритмы их компьютерной реализации с использованием Python, включая разработку интерактивного веб-приложения.

\textbf{Структура работы.} Работа состоит из введения, двух глав основной части, заключения, списка из 20 использованных источников и приложения с листингом программного кода. Во \textbf{введении} обоснована актуальность темы, сформулированы цель, задачи, объект и предмет исследования, описана структура работы. \textbf{Глава 1} представляет теоретический обзор: даются определения основных понятий (звук, звуковая волна, частота, гармоники), излагаются математические основы рядов Фурье, рассматриваются направления исследований и литературные источники по анализу звука (включая историческую справку и современные разработки), проводится их критический анализ. \textbf{Глава 2} посвящена методологии и практическому анализу: описаны методы исследования (математического и программного моделирования звуковых волн), процесс сбора и обработки данных (генерация тестовых сигналов, дискретизация); представлены программные реализации --- определение и воспроизведение звуковых волн средствами Python, преобразование математического описания синусоидальной волны в слышимый звук, объединение (суммирование) нескольких волн для получения сложного звука, разложение звуковой волны в ряд Фурье для выявления её частотных составляющих. Также в этой главе описывается разработка и функциональность интерактивного веб-приложения 'fourier-wave-analyzer', созданного для визуализации и экспериментирования с анализом звуковых волн. В этой главе приводятся результаты вычислительных экспериментов и примеры кода. В \textbf{заключении} подведены итоги, сформулированы выводы о возможностях и ограничениях методов анализа звука с помощью рядов Фурье, а также намечены перспективы дальнейших исследований (в частности, применение быстрого преобразования Фурье, анализ реальных звуковых сигналов, улучшение качества синтеза звука и др.).

\section{Теоретический обзор}

\subsection{Основные понятия звуковых волн и гармонического анализа}

Звук физически представляет собой механическую волну --- колебания давления в среде (например, в воздухе), распространяющиеся в пространстве. Эти колебания, достигая барабанной перепонки уха, вызывают её вибрацию, что воспринимается мозгом как звук определённой высоты и громкости\cite{fletcher}, \cite{rossing}. В общем случае звуковая волна --- это непрерывная функция $x(t)$, описывающая изменение давления (или перемещения частиц среды) во времени $t$. Если источник звука генерирует тон устойчивой высоты (например, камертон или монотонно звучащая струна), то установившаяся звуковая волна близка к \textit{гармоническому колебанию}, то есть синусоидальной форме:
\[
x(t) = A \sin(2\pi f t + \varphi),
\]
где $A$ --- амплитуда, $f$ --- частота (Гц), $\varphi$ --- начальная фаза. Такая синусоидальная волна соответствует \textit{чистому тону} (музыкальному звуку без обертонов). Чистый тон в природе встречается редко (пример --- звук камертона); большинство звуков являются \textit{сложными}, то есть состоят из нескольких синусоидальных компонентов разных частот и амплитуд \cite{fletcher}.

Если звуковая волна носит периодический характер (например, музыкальная нота, играемая на инструменте, имеет почти периодическую волну, повторяющуюся с периодом $T$), то такую функцию $x(t)$ можно представить в виде ряда Фурье --- суммы простых гармонических функций. \textbf{Ряд Фурье} для периодической функции $x(t)$ с периодом $T$ имеет вид\cite{opp-dsp}:
\[
x(t) = \frac{a_0}{2} + \sum_{n=1}^{\infty} \left( a_n \cos \frac{2\pi n}{T}t + b_n \sin \frac{2\pi n}{T}t \right),
\]
где коэффициенты $a_n$ и $b_n$ определяются интегралами:
\[
a_n = \frac{2}{T} \int_{0}^{T} x(t)\cos\frac{2\pi n}{T}t\,dt, \quad 
b_n = \frac{2}{T} \int_{0}^{T} x(t)\sin\frac{2\pi n}{T}t\,dt \quad (n \ge 1),
\]
а $a_0 = \frac{2}{T} \int_{0}^{T} x(t)\,dt$ --- постоянная составляющая (сдвиг среднего уровня функции). Данная формула показывает, что любой периодический сигнал может быть разложен на набор синусоид с частотами, кратными основной (фундаментальной) частоте $f_1 = 1/T$. Эти синусоиды называются \textit{гармониками} или гармоническими компонентами. Коэффициенты $a_n$ и $b_n$ являются весами (амплитудами) косинусных и синусных составляющих, характеризуя вклад каждой гармоники в общий сигнал. Совокупность всех гармоник и их амплитуд называется \textit{спектром} сигнала.

Важно отметить, что разложение в ряд Фурье \textbf{обратимо}: по всем коэффициентам $a_n, b_n$ можно полностью восстановить исходную функцию $x(t)$\cite{opp-dsp}. Таким образом, между временным представлением сигнала (волна $x(t)$) и частотным представлением (набор гармоник) существует взаимнооднозначное соответствие. Переход к представлению сигнала через гармоники называется \textit{гармоническим (спектральным) анализом}.

Понятие спектрального анализа звука было впервые научно обосновано в XIX веке. Немецкий учёный Г. Гельмгольц в работе «Учение о слуховых ощущениях» (1863 г.) экспериментально показал, что любое сложное музыкальное звучание можно представить как наложение чистых тонов, а различие тембров объясняется разным составом гармоник\cite{helmholtz}. Фактически, Гельмгольц подтвердил применимость математических рядов Фурье к анализу звука, став основоположником физиологической акустики. Ещё ранее (1822 г.) Ж. Фурье, исследуя распределение тепла, открыл метод разложения функций в тригонометрические ряды; позднее был осознан и его общий физический смысл: согласно \textit{акустическому закону} Г. С. Ома (1843 г.), человеческое ухо воспринимает только амплитуды гармонических составляющих сложного звука, но не их фазы\cite{opp-dsp}. Это объясняет, почему спектр (набор частот и амплитуд) определяет тембр.

Таким образом, с точки зрения гармонического анализа, любой \textbf{музыкальный звук} с определённой высотой является периодической звуковой волной, основной (фундаментальной) частоте $f_1$ которой соответствует воспринимаемая высота тона (например, $f_1=440$~Гц --- нота Ля первой октавы). Кроме основного тона, присутствуют гармоники с частотами $f_2=2f_1$, $f_3=3f_1$, и т.д. Их амплитуды обычно убывают с ростом номера гармоники, однако характер этого убывания и наличие/отсутствие некоторых гармоник индивидуально для каждого инструмента. На рисунке ниже приведён пример спектрального состава звука в виде суммы гармоник:
\begin{center}
\includegraphics[width=1\textwidth]{second/Fourier_Series.png}
\captionof{figure}{Иллюстрация приближения квадратной волны суммой гармоник: красная кривая --- сумма первых $N$ синусоидальных гармоник, синяя --- целевой прямоугольный сигнал (квадратурная волна). С увеличением числа гармоник аппроксимация улучшается.}
\end{center}

На приведённой иллюстрации показано, как добавление всё большего числа гармоник (синусоид разной частоты, кратной основной) позволяет получить всё более сложную форму сигнала. Данный пример соответствует разложению \textit{прямоугольной волны} (сигнал, чередующийся между двумя уровнями) в ряд Фурье: для такого сигнала присутствуют только нечётные гармоники с амплитудами, обратно пропорциональными номеру гармоники\cite{opp-dsp}. При суммировании 1, 2, 3,\,\dots\, гармоник (верхние графики) форма волны постепенно приближается к прямоугольной (нижний график), причём для точного воспроизведения идеального скачкообразного сигнала в теории потребовалось бы бесконечно много гармоник (что практически недостижимо, но частоты выше порога слышимости $\sim$20~кГц не влияют на восприятие\cite{opp-dsp}).

Таким образом, ряд Фурье даёт математический и наглядный способ изучения звукового сигнала через его частотное содержание. \textbf{Спектр} периодического звука можно изобразить в виде линейчатой спектральной диаграммы, где по оси частот отложены гармоники (основная $f_1$, вторая $f_2=2f_1$ и т.д.), а по вертикали --- их амплитуды. Например, для флейты спектр содержит сильную основную гармонику и сравнительно слабые высокие гармоники, что делает тембр <<мягким>>, близким к чистому тону; для скрипки или трубы характерны более заметные высокие гармоники, придающие резкость звучанию\cite{fletcher}. Спектральный подход позволяет численно и графически сравнивать тембры различных инструментов, анализировать качество звука. Так, в работе Fletcher \& Rossing (1998) приведены спектры звука флейты и скрипки, обе играют ноту соль (392~Гц): у флейты энергия сосредоточена в первых нескольких гармониках, тогда как у скрипки наблюдается множество гармоник значительной амплитуды \cite{fletcher}. Именно разным соотношением гармоник объясняется различие звучания при одинаковой высоте ноты.

Кроме того, анализ гармоник важен при наложении звуков: если одновременно звучат две ноты, в суммарном сигнале присутствуют две группы гармоник. Наш слух может разложить такой аккорд на отдельные ноты, подобно математическому разложению функции на составляющие.

\subsection{Ряд Фурье и преобразование Фурье в обработке сигналов}

Рассмотренный выше тригонометрический ряд Фурье применяется для \textit{периодических сигналов}. Однако в инженерной практике встречаются и непериодические сигналы, например звуковые фрагменты произвольной длины, шумы, одиночные звуки. Для анализа непериодических сигналов используется \textit{преобразование Фурье} --- обобщение идеи рядов Фурье на случай бесконечного периода. В таком пределе частоты гармоник становятся непрерывно распределёнными, и спектр представляется не дискретными линиями, а непрерывной функцией частоты (спектральной плотностью). Тем не менее, в рамках данной работы рассматриваются в основном устойчивые периодические звуки (музыкальные ноты), для которых достаточно аппарата дискретных рядов Фурье.

% В реальных технических приложениях анализ звука обычно производится с помощью \textbf{быстрого преобразования Фурье (FFT)}. FFT --- это алгоритм для быстрого вычисления спектра по цифровому сигналу. Алгоритм был предложен Кули и Тьюки в 1965 г.\cite{cooley-tukey} и позволяет рассчитывать дискретный спектр ($a_n, b_n$ или эквивалентно комплексные коэффициенты) с вычислительной сложностью $O(N \log N)$ операций вместо $O(N^2)$ для прямого применения формул интеграла. Алгоритм FFT находит применение не только в анализе сигналов, но и в вычислительной математике – например, для быстрого перемножения многочленов посредством свёртки коэффициентов. Благодаря этому FFT является фундаментальным алгоритмом как в цифровой обработке сигналов, так и в алгоритмической арифметике.

В реальных технических приложениях анализ звука обычно производится с помощью \textbf{быстрого преобразования Фурье (FFT)}. Следует подчеркнуть, что дискретное преобразование Фурье (DFT) представляет собой математический метод, позволяющий перейти от временного представления функции (сигнала) к её спектральному представлению. Алгоритм FFT, предложенный Кули и Тьюки в 1965 г.\cite{cooley-tukey}, является эффективной реализацией DFT и позволяет вычислять спектральные характеристики сигнала с вычислительной сложностью $O(N \log N)$, что существенно превосходит прямое применение интегральных формул, требующих $O(N^2)$ операций. Помимо обработки сигналов, быстрое преобразование Фурье нашло широкое применение в вычислительной математике для эффективного умножения многочленов. При представлении многочленов через их значения в точках комплексной плоскости, БПФ позволяет выполнить умножение многочленов высоких степеней за $O(N \log N)$ операций вместо $O(N^2)$, что имеет большое значение в компьютерной алгебре, криптографии и теории кодирования. Этот алгоритмический подход особенно ценен при работе с многочленами большой степени, когда классические методы становятся вычислительно неэффективными.

% Таким образом, FFT является фундаментальным инструментом в цифровой обработке сигналов, обеспечивая высокую эффективность анализа временных рядов и их спектральных характеристик.

% FFT стал повсеместно использоваться в цифровой обработке сигналов и нашёл применение в анализаторах спектра, эквалайзерах, компрессорах аудио и др. 

% Интегральное преобразование Фурье и FFT выходят за рамки непосредственной темы исследования, однако в \textbf{главе 2} будут частично использованы подходы дискретного спектрального анализа при обработке данных на компьютере.

Важным аспектом в обработке звука является \textbf{дискретизация} (оцифровка) сигнала. Чтобы обработать или проанализировать звуковую волну на компьютере, её необходимо представить в виде массива чисел --- значений функции $x(t)$ на последовательности дискретных моментов времени. Частота дискретизации должна быть достаточно высокой, чтобы адекватно представить все гармоники исходного звука. Согласно теореме Котельникова (Найквиста--Шеннона), частота дискретизации должна превышать удвоенную максимальную частоту, присутствующую в сигнале\cite{shannon}. Для аудиосигналов верхняя граница слышимого диапазона $\sim$20~кГц, поэтому стандартом де-факто является частота 44100~Гц (CD-качество), достаточная для точного представления слышимого звука. При этой частоте дискретизации шаг между отсчётами $\Delta t \approx 22.68$ мкс\cite{opp-dsp}. В работе именно эта частота используется при генерации и проигрывании звуков.

Далее, \textbf{квантование по амплитуде}: каждому отсчёту звукового сигнала при оцифровке присваивается некоторое цифровое значение с конечной разрядностью. Широко используется 16-битное представление (целые от -32768 до 32767), обеспечивающее динамический диапазон $\sim$96 дБ. В данной работе для синтезированных сигналов также применяется 16-битный формат, соответствующий параметру \texttt{size=-16} библиотеки PyGame. С увеличением битовой глубины точность представления амплитуды возрастает, уменьшается уровень шумов квантования, однако растёт объём данных.

Подводя итог, математический аппарат рядов Фурье является краеугольным камнем анализа звука. Классические результаты Фурье, Гельмгольца и Ома заложили основу понимания природы музыкального звука через гармоники. Современное развитие вычислительных методов (преобразование Фурье, алгоритмы FFT) и цифровых технологий позволяет эффективно применять спектральный анализ на практике: от анализа звуков музыкальных инструментов до автоматического распознавания речи и музыкальных сигналов.


\section{Методология и анализ звуковых волн}

Для достижения целей работы использован комплекс методов: теоретический анализ (математическое разложение функций в ряд Фурье) и вычислительный эксперимент (генерация и обработка сигналов средствами программирования). Сначала выбрана модель звукового сигнала, удобная для исследований. В качестве таковой рассматриваются синтетически сгенерированные сигналы --- это позволяет точно знать их структуру (например, частоты, входящие в состав сигнала) и сравнивать с результатами анализа. Были сгенерированы:
\begin{itemize}
    \item Простые гармонические колебания с заданной частотой --- для проверки возможности воспроизведения синусоидального звука и корректности его восприятия.
    \item Суммы двух гармонических колебаний --- для моделирования простейших аккордов и изучения их временных и спектральных характеристик.
    \item Сложные периодические сигналы, составленные из нескольких гармоник (например, сигнал прямоугольной формы, аппроксимированный частичным рядом Фурье) --- для тестирования разложения сложного сигнала на исходные компоненты.
    \item Случайные шумовые сигналы --- в качестве контроля и для демонстрации отличия упорядоченного (гармоничного) сигнала от шума.
\end{itemize}

Таким образом, объектом экспериментального исследования являются не реальные записанные звуки, а синтетические сигналы, что оправдано целью работы (проверка методов Фурье-анализа) и ограниченностью технических средств (воспроизведение синтезированного звука возможно сразу из программы, тогда как анализ реального звука потребовал бы считывания файла или подключения микрофона). Тем не менее, принципы, проверенные на синтетических примерах, в дальнейшем могут быть перенесены на анализ реальных звуковых данных.

Инструментом исследования выбран язык \textbf{Python} версии 3 с рядом библиотек:
\begin{itemize}
    \item \textbf{NumPy} --- для работы с массивами данных (представления дискретных сигналов и вычисления над ними);
    \item \textbf{Matplotlib} --- для построения графиков сигналов (амплитуда от времени, спектры и т.д.);
    \item \textbf{PyGame} --- для воспроизведения звука из массива сэмплов. Библиотека PyGame предоставляет модуль \texttt{pygame.sndarray}, позволяющий передать массив \texttt{numpy} напрямую аудиоустройству, минуя необходимость сохранения во внешний аудиофайл.
\end{itemize}

Частота дискретизации сигналов в экспериментах принята $F_s = 44100$ Гц. Битовая глубина звука установлена 16 бит (со знаком, целое), как это принято в аудио-CD. Эти параметры явно задаются при инициализации звуковой системы PyGame. Сами сигналы хранятся в массивах NumPy типа \texttt{int16} для соответствия битовой глубине.

Для каждого эксперимента был реализован определённый \textbf{алгоритм}:
\begin{enumerate}
    \item Генерация массива отсчётов звукового сигнала $x[n]$ длиной $N = F_s \cdot T$ (где $T$ --- требуемая длительность звука в секундах). Генерация могла производиться случайно (для шума) или по явной формуле $x[n] = f(t_n)$, где $t_n = n/F_s$ --- время $n$-го отсчёта, а $f(t)$ --- заданная функция (например, $A\sin(2\pi f t)$).
    \item Воспроизведение полученного сигнала с помощью \texttt{pygame.sndarray.make\_sound} и метода \texttt{Sound.play()}. На этом этапе осуществлялась проверка на слух: соответствует ли характер звучания ожидаемому (гармонический тон, сумма тонов, шум и т.д.).
    \item Построение графика временной реализации сигнала $x(t)$ для визуального анализа формы волны.
    \item Вычисление коэффициентов ряда Фурье для полученного дискретного сигнала. Здесь для периодических сигналов длительности $T$ предполагается, что фрагмент представляет ровно один период (или целое число периодов) сигнала. Тогда применимы формулы для $a_n, b_n$, где интегралы заменяются на суммы. В некоторых экспериментах, чтобы упростить расчёты, бралась длительность, равная целому числу периодов (например, 1 секунда сигнала при частоте 5 Гц содержит ровно 5 периодов); в других случаях использован прямой расчёт через интеграл (с последующим усреднением ошибок дискретизации).
    \item Анализ полученных спектральных коэффициентов: сравнение с ожидаемыми (например, известно, что сумма двух синусоид должна иметь спектр с двумя гармониками --- проверяется, действительно ли остальные коэффициенты близки к нулю). На практике, вместо вычисления интегралов, нередко использовался встроенный \texttt{numpy.fft}, чтобы убедиться в правильности результатов (FFT давал комплексные коэффициенты, модуль которых сравнивался с рассчитанными амплитудами гармоник).
    \item Дополнительно, реализация обратной задачи: по вычисленным (или теоретически заданным) коэффициентам ряда Фурье восстановление сигнала и сравнение с исходными. Это служит для проверки полноты разложения.
\end{enumerate}

Следуя этой методике, проведён ряд экспериментов, результаты которых приведены в следующих разделах. Общий план главы 2 таков: сначала демонстрируется программное определение и воспроизведение элементарных звуковых волн (белый шум, синусоида), затем рассматривается формирование более сложных волн сложением, и наконец --- обратная задача, разложение заданной сложной волны на синусоиды (ряд Фурье). 

\subsection{Воспроизведение звуковых волн в Python}

В данном разделе рассматривается процесс генерации и воспроизведения звуковых волн в~Python, что позволяет переводить математические функции, описывающие волну во времени, в реальный аудиосигнал. Мы будем ориентироваться на стандартную для аудио-CD частоту дискретизации, равную $44100$~Гц (то есть 44100 отсчётов в секунду) и использовать 16-битные целые числа для хранения амплитуды каждого отсчёта. Подобное представление даёт достаточно высокое качество звука и при этом хорошо согласуется с библиотекой \texttt{PyGame}, которая способна интерпретировать одномерные массивы данных как аудиопоток.

На рисунке~\ref{fig:wave_schema} схематично показано, как функция давления воздуха $f(t)$, определённая во времени, преобразуется к массиву дискретных значений, который затем воспроизводится с помощью программных инструментов. Каждое число в массиве отвечает некоторому фиксированному моменту времени, а величина этого числа задаёт амплитуду (или интенсивность) сигнала, подаваемого на динамик.

\begin{figure}[ht!]
\centering
\includegraphics[width=1\textwidth]{second/wave_schema.png}
\caption{Принцип перехода от непрерывной функции звуковой волны к набору дискретных отсчётов.}
\label{fig:wave_schema}
\end{figure}

\subsubsection{Воспроизведение первого звука}
\label{sec:first_sound}

На начальном этапе полезно проверить механизмы воспроизведения на примере \textit{белого шума}. Белый шум представляет собой сигнал, в котором отсчёты равномерно распределены по некоторому диапазону значений и не имеют периодической структуры. Для создания белого шума достаточно сгенерировать массив случайных целых чисел в заданном диапазоне (например, от $-10000$ до $+10000$), где каждая точка соответствует мгновенному значению волны. Полученный набор при воспроизведении звучит как «шипение», аналогичное фону не настроенного радиоприёмника. Полный вариант функции, включая возможные проверки и дополнительную логику, приведён в~\textbf{Приложении А}.

На рисунке~\ref{fig:noise_100} приведён пример графика, содержащего первые 100 выборок массива случайных чисел. Видно, что значения распределены хаотично, без явно выраженных закономерностей. Если построить аналогичные графики для, скажем, 441 точек, в соответствии с рисуноком~\ref{fig:noise_441}, то можно увидеть масштабирование по времени, однако природа шума останется прежней --- отсутствие периодичности и монотонности.

\begin{figure}[h!]
\centering
\includegraphics[width=1\textwidth]{second/noise-100.png}
\caption{Белый шум: пример первых 100 отсчётов случайных значений.}
\label{fig:noise_100}
\end{figure}

\begin{figure}[h!]
\centering
\includegraphics[width=1\textwidth]{second/noise-441.png}
\caption{Белый шум: первые 441 отсчёт, соответствующие примерно $0{,}01$~сигналы при 44100~Гц.}
\label{fig:noise_441}
\end{figure}

С точки зрения амплитуды звука, чем больше модуль чисел в массиве, тем громче результат при воспроизведении. Если, к примеру, использовать диапазон от $-32768$ до $+32767$, звук будет максимально возможной громкости (для 16-битного формата), тогда как меньшее числовое окно, скажем $-5000..+5000$, приведёт к более тихому звучанию. Хотя белый шум и не представляет собой музыкальную ноту, он хорошо демонстрирует общий подход: передача массива из $44100$ (или иного подходящего количества) целочисленных значений в аудиосистему \texttt{PyGame} даёт секунду непрерывного звучания.

\subsubsection{Воспроизведение музыкальной ноты}

После воспроизведения шума следующим шагом становится генерация упорядоченных волновых форм, обладающих периодической структурой. Именно такие периодические сигналы воспринимаются человеческим ухом как музыкальные ноты. Классический пример --- \textbf{прямоугольная волна}, в которой отсчёты последовательно принимают два противоположных значения (скажем, $+A$ и $-A$), переходя от одного уровня к другому через равные интервалы времени.

На рисунке~\ref{fig:rect_part} схематично показан один период прямоугольной волны, состоящий из 50 повторений $+A$ и 50 повторений $-A$. Далее этот блок в 100 точек может многократно повторяться, формируя сигнал длиной 44100 выборок за секунду. Если мы хотим получить частоту, близкую к 440--441~Гц (нота ля, A4), достаточно повторить такой блок ровно 441 раз. Аналогично можно рассчитать количество повторов и длину каждого периода для других музыкальных частот (например, для $\nu=350$~Гц или $\nu=523.25$~Гц).

\begin{figure}[h!]
\centering
\includegraphics[width=1\textwidth]{second/rect_100.png}
\caption{Короткий период прямоугольной волны из 100 точек: 50~раз $+A$ и 50~раз $-A$.}
\label{fig:rect_part}
\end{figure}

Если рассмотреть чуть более широкий фрагмент (например, первые 1000 отсчётов итогового массива), становится ясно, что во времени наблюдается повтор этой структурной сотни. На рисунке~\ref{fig:rect_1000} показано, как волна скачкообразно переходит между двумя значениями, что вызывает высокую насыщенность гармоник в спектре и даёт характерный резкий тембр.

\begin{figure}[h!]
\centering
\includegraphics[width=1\textwidth]{second/rect_1000.png}
\caption{Первые 1000 отсчётов из итоговых 44100 прямоугольной волны для одной секунды звучания.}
\label{fig:rect_1000}
\end{figure}

Альтернативой прямоугольной форме является \textbf{синусоидальная} волна, более близкая к реальным физическим колебаниям (например, колебания струны или голосовых связок). Синусоида задаётся формулой
\[
f(t) \;=\; A \,\sin(2\pi \,\nu \, t),
\]
где $\nu$ --- частота (в герцах), а $A$ --- амплитуда. При воспроизведении такой волны можно наблюдать более мягкий и «чистый» звук по сравнению с резким прямоугольным сигналом. Если за одну секунду проходим ровно $\nu$ полных циклов синуса, то человеческое ухо воспринимает ноту высотой $\nu$. 

% На рисунке~\ref{fig:sine_half} приведён участок синусоидального сигнала, отражающий плавный переход от положительных к отрицательным значениям и обратно --- такая форма лишена резких скачков, что влияет на тембр звучания.

% \begin{figure}[h!]
% \centering
% \includegraphics[width=1\textwidth]{second/sine_half.png}
% \caption{Участок синусоидальной волны: плавное нарастание и убывание амплитуды.}
% \label{fig:sine_half}
% \end{figure}

Подобно тому, как мы генерировали белый шум или прямоугольную волну, синусоида превращается в массив из 44100 целочисленных отсчётов путём равномерного взятия её значений в моменты $t = 0, \frac{1}{44100}, \frac{2}{44100}, \dots, 1$. Важным параметром остаётся выбранная амплитуда $A$, которая напрямую определяет громкость воспроизведения.

Таким образом, используя базовые средства Python (библиотека \texttt{NumPy} для массивов и функции \texttt{PyGame} для воспроизведения), мы можем легко переходить от аналитических описаний звуковых волн к реальному звучанию. Путём изменения частоты и формы волны (прямоугольной, синусоидальной, треугольной и т.\,д.) достигается широкий спектр эффектов и тембров. Этот процесс важен не только для синтеза музыки, но и для экспериментов с \textit{рядом Фурье}, где каждая периодическая волна может быть разложена на набор простых синусоидальных компонент, каждая из которых соответствует своей гармонике (частоте).

В дальнейшем мы рассмотрим, как комбинировать несколько волн, формируя \textit{аккорды} (сложение двух или более периодических функций) и как математически анализировать полученные сигналы. Но вначале достаточно понять, что простое воспроизведение массива целочисленных значений даёт готовый звуковой результат, а форма этой последовательности напрямую определяет, будет ли это шум, отчётливая нота или что-то среднее между ними.

\subsection{Преобразование синусоидальной волны в звук}

Важным примером периодических сигналов, применяемых для формирования музыкальных нот, являются \textbf{синусоидальные функции} — классические \(\sin\) и \(\cos\). Они обладают фундаментальными свойствами для описания плавных колебаний, часто встречающихся в природе (например, при колебаниях струн музыкальных инструментов, акустических колебаниях в воздухе и т.\,д.). С точки зрения математики синусоида — это функция вида
\[
   f(t) \;=\; A \,\sin\bigl(\omega t + \varphi\bigr),
\]
где \(A\) — \textit{амплитуда}, \(\omega\) — \textit{круговая частота}, а \(\varphi\) — \textit{начальная фаза}. Для удобства при цифровом синтезе музыки часто берут \(\omega = 2\pi \times \nu\), где \(\nu\)~— частота в герцах (Гц). Тогда синусоидальная волна соответствует гармоническому колебанию с повторяющимся периодом
\[
   T \;=\; \frac{1}{\nu}.
\]

В контексте цифрового воспроизведения аудио (например, средствами \texttt{PyGame} в Python) нам нужно превратить аналитически заданную синусоиду во множество (часто 44100) отсчётов за одну секунду — то есть выполнить \textit{дискретизацию}. Каждое число в полученном массиве будет соответствовать мгновенной амплитуде синусоиды в равномерной сетке по времени. Затем этот массив интерпретируется звуковой подсистемой как один канал аудиопотока, дающий непрерывный звук желаемой высоты (определяемой частотой) и громкости (зависящей от амплитуды).

С математической точки зрения \(\sin\) и \(\cos\) — базовые периодические функции с \textbf{периодом} \(2\pi\). Их угловая частота \(\omega\) по умолчанию равна \(1\), что соответствует одной полной «волне» на отрезке \(0 \le t \le 2\pi\). Удобно масштабировать аргумент функции, чтобы добиваться разных частот. Если нам нужна синусоида с частотой \(\nu\)~Гц, то достаточно взять
\[
   f(t) \;=\; A\,\sin\bigl(2\pi\,\nu \,t\bigr).
\]
Тогда за время \(1\)~секунда функция успеет сделать \(\nu\) полных колебаний. Параметр \(A\) напрямую определяет \textit{амплитуду} и, следовательно, воспринимаемую громкость сигнала.

На рисунке~\ref{fig:sin_2pi} схематично показана функция \(\sin(t)\), повторяющаяся каждые \(2\pi\) единиц аргумента. Модификации вида \(\sin(\alpha \, t)\) сдвигают частоту (то есть изменяют период), а умножение на константу увеличивает или уменьшает амплитуду.

\begin{figure}[h!]
\centering
\includegraphics[width=1\textwidth]{second/sin_2pi.png}
\caption{Схематический график \(\sin(t)\), имеющий период \(2\pi\).}
\label{fig:sin_2pi}
\end{figure}

% \subsubsection{Удобное формирование синусоиды желаемой частоты}

% Частоту \(\nu\) часто берут равной высоте музыкальной ноты. Например, нота ля (A4) обычно соответствует \(\nu = 440\,\text{Гц}\). Для работы в 16-битном формате при 44100 выборках в секунду мы можем написать функцию, которая генерирует отсчёты вида
% \[
%    A \,\sin\bigl(2\pi\,\nu\,t_i\bigr),\quad
%    t_i = \frac{i}{44100}, \quad i=0,1,\dots,44099.
% \]
% Чтобы код оставался гибким, удобно ввести параметр, отвечающий за \textit{амплитуду} (\texttt{amplitude}) и за \textit{частоту} (\texttt{frequency}). Таким образом, для частоты \(\nu = 441\)~Гц и амплитуды порядка 8000 мы получим синусоиду, ясно слышимую как тон, близкий к \(\nu=440\)~Гц.

% \begin{figure}[h!]
% \centering
% \includegraphics[width=1\textwidth]{second/sine_stretch.png}
% \caption{Пример синусоиды, растянутой по вертикали на амплитуду и сжимаемой по горизонтали для нужной частоты.}
% \label{fig:sine_stretch}
% \end{figure}

\subsubsection{Дискретизация и воспроизведение синусоиды}

Важным этапом превращения аналитически заданной синусоиды в реальный звук является \textbf{дискретизация}, которая состоит в выборе фиксированного набора временных отсчётов и последующем переводе значений волны в 16-битные целые числа. Если требуется, скажем, одна секунда звучания, то при стандартной частоте дискретизации $44100$~Гц нужно вычислить 44100 отсчётов. Каждый отсчёт соответствует значению синусоидальной функции 
\[
   f(t) \;=\; \text{amplitude}\,\times \sin\bigl(2\pi \,\text{frequency}\times t\bigr),
\]
где $t$ --- конкретный момент времени, выраженный в секундах.

\begin{enumerate}
\item \textbf{Задать} параметры: \texttt{frequency} (Гц), \texttt{amplitude}, \texttt{duration}, \texttt{sample\_rate}.
\item \textbf{Для} $i=0,1,\dots,\texttt{sample\_rate}\times\texttt{duration}-1$:
  \begin{itemize}
    \item Вычислить $t_i = i / \texttt{sample\_rate}$.
    \item Определить $v_i = \text{amplitude} \times \sin\bigl(2\pi\,\text{frequency}\times t_i\bigr)$.
    \item Привести $v_i$ к целому типу в диапазоне $\bigl[-32768,\,+32767\bigr]$ (16-бит).
  \end{itemize}
\item \textbf{Сформировать} одномерный массив отсчётов.
\item \textbf{Вызвать} функцию аудиосистемы (например, \texttt{PyGame}), которая воспримет массив как звуковой поток.
\end{enumerate}

\noindent
Ниже приведён фрагмент кода, который иллюстрирует этот процесс:

\begin{verbatim}
sine_wave = A * np.sin(2 * math.pi * f * n / 44100)
sine_wave = sine_wave.astype(np.int16)

tone = pygame.sndarray.make_sound(sine_wave)
tone.play()
\end{verbatim}

Таким образом, данный код генерирует одномерный NumPy-массив с нужной длиной, значения которого соответствуют дискретным точкам синусоиды. Затем массив передаётся в функцию \texttt{pygame.sndarray.make\_sound}, и результат автоматически проигрывается через звуковую подсистему. Параметры, такие как \texttt{frequency} и \texttt{amplitude}, позволяют гибко управлять высотой ноты и громкостью. Полный вариант функции, включая возможные проверки и дополнительную логику, приведён в~\textbf{Приложении Б}.


При проигрывании такого массива мы получаем непрерывный звук с заданными параметрами. В частности, если брать \(\texttt{duration}=1\)~с, то массив из 44100 точек воспроизводится за одну секунду.

Таким образом, \textbf{синусоидальная} волна является базовым строительным блоком, из которого, пользуясь принципами разложения в ряды Фурье, можно получить практически любую другую периодическую форму, вплоть до прямоугольных и пилообразных сигналов. Однако в своём «чистом» виде синус даёт плавное и мягкое звучание, часто ассоциируемое с флейтой или другими «чистыми» тонами без обертонов.

\subsection{Объединение звуковых волн}

Объединение звуковых волн даёт фундаментальный пример суперпозиции сигналов и позволяет наглядно продемонстрировать, как различные звуки или ноты могут взаимодействовать друг с другом при сложении. Данный процесс представляет интерес как с прикладной, так и с теоретической точки зрения: в одном случае удобно видеть, как складываются музыкальные ноты, а в другом --- как синусоидальные функции образуют более сложные формы волн. В рамках цифровой обработки звука (DSP) к процедуре суммирования сводится множество операций: наложение нескольких инструментов, генерация аккордов, моделирование биений при близких частотах и построение произвольных сигналов из базиса гармоник \cite{tolstov}.

Важной целью раздела является демонстрация того, как математические операции над функциями (в частности, их сложение и умножение на скаляр) переносятся на модель звуковых волн, а также как суперпозиция синусоид ведёт к богатству возникающих тембров и музыкальных созвучий. В работе опираются на фундаментальные свойства рядов Фурье, согласно которым любой периодический сигнал можно разложить в сумму синусоид и косинусоид соответствующих частот, что даёт общий метод анализа и синтеза волновых форм \cite{bracewell}.

\subsubsection{Сложение выборок звуковых волн}
При цифровом представлении звука непрерывная функция давления воздуха во времени заменяется набором дискретных отсчётов, или \emph{выборкой} (англ. sample). Если заданы два звуковых сигнала, каждый из которых представлен массивом целых (или вещественных) чисел одинаковой длины, то суммирование этих массивов поэлементно соответствует наложению звуковых волн. В компьютерных средах (например, в Python с пакетом NumPy) такое сложение выполняется за один оператор сложения над массивами. Результирующий массив отражает комбинированный сигнал, который при воспроизведении воспринимается как одновременное звучание двух исходных волн. Полный вариант функции, включая возможные проверки и дополнительную логику, приведён в~\textbf{Приложении В}.

Если каждый из исходных сигналов соответствует отдельной ноте, то их суммарная волна представляет собой аккорд. В случае, когда частоты исходных волн близки, возможно возникновение биений --- характерных пульсаций, связанных с чередованием фазовой синхронизации и противофазного наложения. С физической точки зрения это описывается как явление интерференции: в моменты, когда пики обеих волн совпадают, возникает конструктивная интерференция (громкость выше), а при противофазе --- деструктивная, приводящая к ослаблению или даже почти полному гашению итогового сигнала на коротком участке.

% \begin{figure}[h!]
% \centering
% \includegraphics[width=1\textwidth]{second/two_signals_sum.png}
% \caption{Условная схема суммирования двух волновых форм, каждая из которых представлена своей выборкой}
% \label{fig:two_signals_sum}
% \end{figure}

% Для наглядности на рисунке~\ref{fig:two_signals_sum} продемонстрирован принцип сложения двух волн во временной области. Человеческое ухо различает итоговую волну как единый звук, обладающий характерной суммарной амплитудой и временной формой.

\subsubsection{Изображение графика суммы двух синусоид}
Рассмотрим наиболее наглядный пример, когда обе звуковые волны являются синусоидами с разными частотами:
\[
f(t) = A_1 \sin(2\pi \nu_1 t), 
\quad
g(t) = A_2 \sin(2\pi \nu_2 t).
\]
Сумма
\[
h(t) = f(t) + g(t) = A_1 \sin(2\pi \nu_1 t) + A_2 \sin(2\pi \nu_2 t)
\]
даёт результирующий сигнал, в котором прослеживается характерная картина интерференции. При численной реализации отрезок времени \(0 \le t \le T\) разбивается на \(N\) точек, и для каждого узла сетки \(t_k = kT/N\) вычисляются значения синусоид с последующим суммированием:
\[
h(t_k) = A_1 \sin(2\pi \nu_1 t_k) \;+\; A_2 \sin(2\pi \nu_2 t_k).
\]
Полученный набор \(\{h(t_k)\}\) можно визуализировать в виде ломаной или непрерывной кривой, что позволяет отслеживать моменты совпадения фаз сигналов и их противофазного сложения.

\begin{figure}[h!]
\centering
\includegraphics[width=1\textwidth]{second/sum_sine_simple.png}
\caption{Пример суммирования двух синусоид (верхние графики) и результирующий сигнал (нижний график)}
\label{fig:sum_sine_signals}
\end{figure}

На рисунке~\ref{fig:sum_sine_signals} схематично показано, как две компоненты (сверху) образуют новую волну (снизу). Временами пики совпадают и происходит усиление (конструктивная интерференция), а временами --- гашение, если волны находятся в противофазе (деструктивная интерференция). При восприятии на слух это может создавать специфические модуляции громкости, если частотные значения \(\nu_1\) и \(\nu_2\) близки.

\subsubsection{Построение линейной комбинации синусоид}
Линейная алгебра позволяет интерпретировать функции как векторы в пространстве сигналов, а операцию сложения --- как векторную операцию. Если задан набор синусоидальных функций
\[
f_1(t), \quad f_2(t), \quad \dots, \quad f_n(t),
\]
то линейная комбинация
\[
F(t) \;=\; \alpha_1 f_1(t) \;+\; \alpha_2 f_2(t) \;+\;\dots\;+\;\alpha_n f_n(t)
\]
даёт новый сигнал, содержащий спектральные компоненты с частотами исходных функций. Параметры \(\alpha_i\) задают амплитудный вклад соответствующей гармоники.

При цифровой реализации каждая функция дискретизируется. Значения
\[
f_i(t_k) = f_i\!\Bigl(\frac{k}{F_s}\Bigr), \quad k = 0,1,\dots,N-1,
\]
где \(F_s\) --- частота дискретизации, складываются по каждому индексу \(k\) с весом \(\alpha_i\). Результатом является итоговая выборка:
\[
F(t_k) = \sum_{i=1}^{n} \alpha_i f_i(t_k).
\]
Такой подход используется в синтезе звуков, поскольку позволяет конструировать разную тембровую палитру, изменяя спектральные коэффициенты.

\begin{figure}[h!]
\centering
\includegraphics[width=0.9\textwidth]{second/lin_combination.png}
\caption{Линейная комбинация нескольких синусоид разной частоты даёт итоговый сигнал сложной формы}
\label{fig:lin_combination_sinus}
\end{figure}

На рисунке~\ref{fig:lin_combination_sinus} приведена условная иллюстрация, показывающая три синусоиды (вверху) и их результирующую сумму (внизу). Видно, что небольшое число простых функций, складываясь, способно формировать волны со значительно более изменчивой формой.

\subsubsection{Построение прямоугольной волны через сумму синусоид}
Интересно, что даже разрывные функции, например прямоугольная волна, могут быть аппроксимированы (приближены) путём суммирования конечного числа синусоидальных функций с правильно подобранными амплитудами и частотами. Бесконечный ряд Фурье, строго говоря, в точности восстанавливает любую периодическую функцию, но даже ограниченное число членов ряда нередко даёт хорошее приближение.

Прямоугольная волна амплитуды \(\pm1\), меняющая знак каждые полпериода, по теореме Фурье разлагается в ряд, содержащий только нечетные гармоники. Иными словами, при её аппроксимации коэффициенты при чётных частотах синусоид равны нулю, а для нечетных гармоник коэффициенты задаются формулой \(\frac{4}{\pi(2n-1)}\), где \(n = 1, 2, 3, ...\). Чем большее количество гармоник взять, тем более «квадратной» станет результирующая форма.

\begin{figure}[h!]
\centering
\includegraphics[width=1\textwidth]{second/square_approx.png}
\caption{Аппроксимация прямоугольной волны конечной суммой синусоид, при увеличении числа гармоник форма становится ближе к ступенчатой}
\label{fig:square_approx}
\end{figure}

На рисунке~\ref{fig:square_approx} условно показано, как рост количества гармоник сглаживает переходы между уровнями \(-1\) и \(+1\). В реальной цифровой системе эти скачки никогда не будут идеальными, но при достаточном количестве членов звук воспринимается как классическая «прямоугольная волна».

При сложении нескольких звуковых волн формируется новый сигнал, обладающий суммарной формой, которая в каждый момент времени является результатом алгебраической суперпозиции значений исходных сигналов. На уровне отдельных примеров показано, что сложение двух синусоидальных волн приводит к чередованию участков конструктивной и деструктивной интерференции, что иллюстрирует появление биений или аккордовых эффектов. С позиций линейной алгебры функция, описывающая результат, является линейной комбинацией исходных функций, а следовательно, при соответствующей выборке параметров можно синтезировать достаточно произвольные формы волн. Наиболее наглядный пример --- приближение прямоугольной волны суммой синусоид нечетных гармоник, соответствующее ряду Фурье для заданной периодической функции.

Таким образом, операции над функциями, рассматриваемыми как волновые сигналы, непосредственно переносятся на операции со звуком в цифровой обработке. Идеи суммирования волн лежат в основе синтеза музыкальных тембров, создания сложных сигнальных паттернов и анализа спектрального состава аудиоданных. Во многих инженерных и научных задачах этот подход позволяет как синтезировать желаемый звук, так и анализировать уже существующие сигналы, разлагая их по базису гармоник.

\subsection{Разложение звуковых сигналов в ряд Фурье}

Анализ и разложение звуковых сигналов на гармонические составляющие занимает центральное место в современной теории обработки сигналов, а также в общей теории функций. Одним из важнейших инструментов такого анализа является классический ряд Фурье, который позволяет представить любую периодическую функцию (в том числе описывающую звуковую волну) в виде линейной комбинации (суперпозиции) базовых гармоник — синусоид и косинусоид различных частот. Данный аппарат широко используется в цифровой обработке звука, сжатии аудио (MP3, AAC), а также в различных физических и инженерных приложениях \cite{tolstov}. 

В данном разделе рассматривается общая схема разложения звуковой волны в ряд Фурье и приводятся основные формулы для вычисления соответствующих коэффициентов. Базовые принципы иллюстрируются примером прямоугольной волны, дополнительно рассматриваются пилообразные и треугольные сигналы. С точки зрения линейной алгебры, звуковые волны можно интерпретировать как векторы из бесконечномерного пространства функций, а синусоиды — как набор ортонормированных базисных векторов. С этой позицией удобно изучать принцип разложения на компоненты и определять, какие именно гармоники входят в состав конкретного звукового сигнала. Так, прямоугольная волна имеет составляющую длины $4/\pi$ в направлении $\sin(2\pi t)$ и составляющую длины $4/3\pi$ в направлении $\sin(6\pi t)$, что соответствует первым двум координатам прямоугольной волны в базисе гармонических функций.

\begin{figure}[h!]
  \centering
  \includegraphics[width=1\textwidth]{second/decomposition_wave.png}
  \caption{Схематичное представление разложения звукового сигнала на гармонические составляющие}
  \label{fig:fourier_illustration}
\end{figure}

На рисунке \ref{fig:fourier_illustration} иллюстрируется идея того, как сложная периодическая волна может рассматриваться как сумма более простых колебаний синусоидальной формы (гармоник). Ниже приводится формальная математика, лежащая за данной идеей.

\subsubsection{Поиск компонент вектора с помощью внутреннего произведения}

Для начала стоит рассмотреть классическую задачу в конечномерном пространстве: пусть имеется вектор 
\[
\mathbf{v} = (v_1, v_2, v_3) \quad \text{в } \mathbb{R}^3.
\]
Если мы хотим найти «проекции» этого вектора на стандартный базис \(\mathbf{e}_1 = (1,0,0)\), \(\mathbf{e}_2 = (0,1,0)\), \(\mathbf{e}_3 = (0,0,1)\), то достаточно воспользоваться скалярным произведением:
\[
\mathbf{v} \cdot \mathbf{e}_1 = v_1, \quad 
\mathbf{v} \cdot \mathbf{e}_2 = v_2, \quad
\mathbf{v} \cdot \mathbf{e}_3 = v_3.
\]
Все остальные компоненты (например, в направлениях отличных от \(\mathbf{e}_i\)) здесь не нужны. Это простое наблюдение обосновывает идею, что \textit{внутреннее произведение} в пространстве векторов — удобный способ вычислять «координаты», или проекции, на заданный набор базисных направлений.

При работе со звуковыми волнами мы имеем дело уже не с трёхмерными векторами, а с функциями \(f(t)\). Чтобы применить аналогичные идеи поиска «проекций», необходимо ввести понятие внутреннего (скалярного) произведения для функций. Пусть \(f\) и \(g\) — две периодические функции с периодом 1, определённые на отрезке \([0,1]\). Определим «весовое» (или интегральное) произведение следующим образом:
\[
\langle f, g \rangle = 2 \int_{0}^{1} f(t)\, g(t)\, dt.
\]
Дополнительный множитель 2 (по сравнению с классическим определением Фурье-рядов) здесь вводится с той целью, чтобы дальнейшие формулы для коэффициентов в ряде Фурье приняли более простую и удобную форму, согласованную с «нормировкой» базовых функций \(\sin(2\pi n t)\) и \(\cos(2\pi n t)\). Более общие подходы могут иметь отличные от 2 коэффициенты, но принятая здесь норма достаточно типична \cite{bracewell}.

Если при вычислении \(\langle f, g\rangle\) получается 0, то говорят, что функции \(f\) и \(g\) \textit{ортогональны}. Аналогичным образом определяется и «длина» функции \(f\):
\[
\| f\|^2 = \langle f, f\rangle.
\]
С учётом приведённого определения и ряда математических фактов, оказывается, что синусоиды и косинусоиды разных частот являются ортонормированным базисом для класса достаточно гладких (или квадратично интегрируемых) периодических функций на \([0,1]\) \cite{opp-sigsys}.

\subsubsection{Определение внутреннего произведения периодических функций}

Для наглядности можно показать, как в численном виде оценивать внутреннее произведение. Пусть функция \(f(t)\) «задана» выборкой значений \(f(t_k)\) на равномерной сетке \(t_k = k\Delta t\) при \(k = 0, 1, \ldots, N-1\) и \(\Delta t = \frac{1}{N}\). Аналогично, предположим, что функция \(g(t)\) также задана набором значений \(g(t_k)\) на той же сетке. Тогда интеграл
\[
\int_{0}^{1} f(t) g(t) \, dt
\]
можно аппроксимировать \textit{суммой Римана}:
\[
\int_{0}^{1} f(t) g(t) \, dt 
\;\approx\; 
\sum_{k=0}^{N-1} f(t_k) g(t_k) \, \Delta t.
\]

% Графическое представление этой схемы суммирования Римана для численного вычисления внутреннего произведения можно увидеть на рисунке~\ref{fig:inner_product_approx}, где наглядно показано, как произведения значений функций в отдельных точках формируют общую аппроксимацию интеграла.

Следует отметить, что для ортонормированного базиса синусоид и косинусоид различных частот, как было указано выше, используется нормировочный множитель 2 в определении внутреннего произведения для периодических функций на интервале [0, 1]. Это связано с «нормировкой» базовых функций \(\sin(2\pi nt)\) и \(\cos(2\pi nt)\). С учётом этого множителя получаем финальную формулу для дискретной аппроксимации:
\[
\langle f, g\rangle \approx 2\sum_{k=0}^{N-1} f(t_k)\,g(t_k)\,\Delta t.
\]
При увеличении \(N\) данная сумма становится всё более точной.

% \begin{figure}[h!]
%   \centering
%   \includegraphics[width=1\textwidth]{second/inner_product_approx.png}
%   \caption{Графическое представление схемы суммирования Римана для численного вычисления \(\langle f, g\rangle\)}
%   \label{fig:inner_product_approx}
% \end{figure}

На практике в ходе цифровой обработки звука подобные суммирования реализуются в алгоритмах быстрого преобразования Фурье (БПФ), которые эффективно вычисляют «скалярные произведения» функции на синусоиды (или косинусоиды) за \(O(N \log N)\) операций \cite{opp-sigsys}.

\subsubsection{Определение функции для поиска коэффициентов Фурье}

Пусть имеется функция \(f(t)\) с периодом 1, которую мы хотим разложить в ряд Фурье. Тогда предполагается, что \(f(t)\) можно представить в виде
\[
f(t) \;=\; 
a_0 \,\phi_{0}(t)
\;+\;
\sum_{n=1}^{N} \Bigl[a_n \cos(2\pi n\,t) + b_n \sin(2\pi n\,t)\Bigr],
\]
где \(\phi_0(t)\) есть некоторая «постоянная» функция, выбираемая для полноты базиса (обычно берут \(\phi_0(t) = \tfrac{1}{\sqrt{2}}\) для нормировки).

Такое представление позволяет искать коэффициенты \(a_0\), \(a_n\) и \(b_n\) по аналогии с тем, как в \(\mathbb{R}^3\) искали координаты вектора при помощи скалярных произведений на базисные векторы. Для \(n \ge 1\):
\[
a_n = \langle f, \cos(2\pi n\,t)\rangle,
\quad
b_n = \langle f, \sin(2\pi n\,t)\rangle,
\]
а для постоянной компоненты
\[
a_0 = \langle f, \phi_0(t)\rangle.
\]
Проверка этих формул напрямую следует из свойств ортонормированного базиса \(\{\phi_0(t), \cos(2\pi n\,t), \sin(2\pi n\,t)\}\), поскольку все указанные функции попарно ортогональны и имеют «единичную» норму.

На языке псевдокода или в Python-подобном синтаксисе можно реализовать процедуру \texttt{fourier\_coefficients}, которая принимает функцию \(f\) и максимальное число гармоник \(N\), а затем выдаёт соответствующие массивы \(\{a_0, a_1, \dots, a_N\}\) и \(\{b_1, \dots, b_N\}\). В упрощённом виде это может выглядеть так:

\begin{verbatim}
def fourier_coefficients(f, N):
    a0 = inner_product(f, const)
    an = [inner_product(f, c(n))
          for n in range(1, N+1)]
    bn = [inner_product(f, s(n))
          for n in range(1, N+1)]
    return a0, an, bn
\end{verbatim}

Для проверки передадим функции \texttt{fourier\_coefficients} известный ряд Фурье и убедимся, что она возвращает известные коэффициенты:

\begin{verbatim}
>>> f = fourier_series(0,[2,3,4],[5,6,7])
>>> fourier_coefficients(f,3)
(-3.812922200197022e-15,
[1.9999999999999887, 2.999999999999999, 4.0],
[5.000000000000002, 6.000000000000001, 7.0000000000000036])
\end{verbatim}

Здесь \(\texttt{integrate\_over\_period}\) — функция численного интегрирования, возвращающая приближённое значение $\int_0^1 f(t) g(t) dt$. Множитель 2 включён, чтобы согласовать обозначения с \(\langle f,g\rangle\), хотя в коде можно было бы напрямую использовать \(\langle f,g\rangle\). \texttt{inner\_product(f, g, N=1000)} -- реализация внутреннего произведения двух функций $f$ и $g$, определенного формулой:
    \[
    \langle f, g \rangle = 2 \int_0^1 f(t) g(t) \, dt
    \]
    Численное интегрирование выполняется методом прямоугольников с $N$ равномерно распределенными точками на интервале от $0$ до $1$. Параметр $N$ по умолчанию равен $1000$, что обеспечивает достаточную точность для большинства практических задач.
     
Полная реализация функции \texttt{fourier\_coefficients} и всех вспомогательных функций, включая \texttt{inner\_product}, \texttt{const}, \texttt{s} и \texttt{c}, приведена в~\textbf{Приложении Г}.

\subsubsection{Поиск коэффициентов Фурье для прямоугольной волны}

В качестве показательного примера рассмотрим \textbf{прямоугольную волну} \( \mathrm{sq}(t)\), принимающую значение +1 при \(0 \le t < 0.5\) (по модулю 1) и -1 при \(0.5 \le t < 1\). Формально:
\[
\mathrm{sq}(t) = 
\begin{cases}
+1, & 0 \le (t \bmod 1) < 0.5,\\
-1, & 0.5 \le (t \bmod 1) < 1.
\end{cases}
\]
Интуитивно, такая волна переключается из +1 в -1 в середине периода и поэтому выглядит «прямоугольной» (прыгает с одной горизонтальной ветви на другую). Разложение в ряд Фурье содержит \textit{только} синусоидальные члены с нечётными номерами:
\[
\mathrm{sq}(t)
=
\frac{4}{\pi} 
\Bigl[\sin(2\pi t) + \frac{1}{3}\sin(6\pi t) + \frac{1}{5}\sin(10\pi t) + \dots\Bigr].
\]
Ряд содержит бесконечно много гармоник вида \(\sin(2\pi(2k+1)\,t)\) с коэффициентами, равными \(\tfrac{4}{\pi}\cdot \tfrac{1}{(2k-1)}\). Отсутствие косинусоид здесь обусловлено чётно-нечётной структурой самой функции \(\mathrm{sq}(t)\) [4]. 

На практике можно взять конечное число гармоник \(N\), например \(N=5\) или \(N=10\), чтобы получить достаточно точное приближение прямоугольной волны. При \(N=5\) получается сумма 5 синусоид, которая уже довольно грубо воспроизводит форму сигнала, а при \(N=50\) функция визуально становится почти «прямоугольной». В точках разрыва амплитуды (при \(t=0.5\)) возникает эффект Гиббса, проявляющийся в виде «колебаний» перед скачком, однако интегральная ошибка всё равно убывает при увеличении \(N\). Полный вариант функции, включая возможные проверки и дополнительную логику, приведён в~\textbf{Приложении Д}.

\begin{figure}[h!]
  \centering
  \includegraphics[width=1\textwidth]{second/rect_fwave_fig.png}
  \caption{Накопление гармоник в ряде Фурье для прямоугольной волны с различными значениями $N$.}
  \label{fig:rect_approx}
\end{figure}

На рисунке~\ref{fig:rect_approx} схематически показано, как при увеличении числа гармоник \(N\) происходит всё более точное приближение прямоугольного сигнала.

\subsubsection{Коэффициенты Фурье для других волнообразных функций}

Подобный подход «складирования» функции из синусоидальных базисных функций работает и для других волнообразных сигналов. Часто на практике встречаются \textit{пилообразные} сигналы (англ. sawtooth wave), \textit{треугольные} (triangle wave) и другие нестандартные периодические формы. Все они могут быть разложены по тому же принципу:
\[
f(t) \;=\; 
a_0 \frac{1}{\sqrt{2}} 
\;+\;
\sum_{n=1}^{\infty} 
a_n \cos(2\pi n\,t) \;+\; 
b_n \sin(2\pi n\,t).
\]
При этом вид диаграммы зависимостей \(a_n\) и \(b_n\) от номера гармоники \(n\) (например, быстрое убывание модуля коэффициентов) описывает, насколько плавно или, наоборот, резко меняется волна во времени. Более плавные и гладкие сигналы обычно имеют быстро затухающие ряды Фурье, а «рваные» функции (с разрывами) сохраняют заметные гармоники до очень высоких частот \cite{downey}.

\begin{figure}[h!]
  \centering
  \includegraphics[width=1\textwidth]{second/1triangle_wave.png}
  \caption{Пример треугольной волны и её разложение на синусоиды и косинусоиды}
  \label{fig:triangle_wave_example}
\end{figure}

На рисунке~\ref{fig:triangle_wave_example} представлен упрощённый вид треугольной волны. Хорошо видно, что чем резче «изломы» на графике исходного сигнала, тем существеннее вклад более высоких гармоник в ряде Фурье. 

Таким образом, разложение звуковой волны в ряд Фурье — это универсальный способ определить амплитуды чистых синусоидальных компонент (гармоник), которые складываются в итоге в наблюдаемый акустический сигнал. С точки зрения музыкальной акустики, данные коэффициенты напрямую связаны с \textit{тембром} звука, ведь тембр определяется отношением амплитуд различных обертонов (гармоник).

Любая периодическая звуковая волна может быть представлена в виде ряда Фурье, то есть набора (возможно, бесконечного) синусоидальных и косинусоидальных слагаемых. Процесс вычисления соответствующих коэффициентов сводится к вычислению «проекций» рассматриваемой функции на гармонические базисные функции, что формально выражается через операцию внутреннего произведения. В итоге модель звука, записанная как суперпозиция гармоник, облегчает дальнейший анализ, фильтрацию и синтез звуковых сигналов. Это даёт теоретическую базу для различных методов спектральной обработки, разработки синтезаторов, анализа музыкальных сигналов и других прикладных задач современной цифровой обработки звука.

Таким образом, разложение в ряд Фурье выступает мощным связующим звеном между линейной алгеброй и теорией функций, а также между математикой и инженерными науками, связанными со звуком. Детальные математические выкладки и вопросы сходимости рядов лежат за рамками краткого обзора, однако они могут быть найдены в классических источниках по анализу Фурье, а также во многих прикладных руководствах по цифровой обработке сигналов \cite{opp-dsp}.

\subsection{Разработка веб-приложения 'fourier-wave-analyzer'}
\label{sec:webapp}

В дополнение к программным экспериментам, описанным выше с использованием скриптов Python, в рамках данной работы было разработано интерактивное веб-приложение 'fourier-wave-analyzer'. Целью создания этого приложения являлась наглядная демонстрация ключевых концепций анализа звуковых волн с помощью рядов Фурье, а также предоставление пользователю возможности самостоятельно экспериментировать с параметрами сигналов.

\textbf{Технологии и архитектура.} Веб-приложение разработано с использованием современных веб-технологий. Бэкенд реализован на языке Python с применением фреймворка Flask (или Django - \textit{уточните используемый фреймворк}), который обрабатывает запросы пользователя, генерирует звуковые данные и выполняет вычисления, связанные с рядами Фурье (используя библиотеки NumPy и, возможно, SciPy). Фронтенд построен с использованием HTML, CSS и JavaScript, обеспечивая интерактивный пользовательский интерфейс. Для визуализации графиков сигналов и их спектров используется библиотека Plotly.js (или Matplotlib, генерирующий изображения на сервере - \textit{уточните библиотеку для графиков}), позволяющая строить динамические и масштабируемые графики прямо в браузере.

\textbf{Функциональные возможности.} Приложение 'fourier-wave-analyzer' предоставляет пользователям следующий основной функционал:
\begin{itemize}
    \item Генерация базовых звуковых волн (например, синусоидальных, прямоугольных, треугольных, пилообразных) с задаваемыми параметрами: частотой, амплитудой, длительностью.
    \item Визуализация временной формы сгенерированной волны.
    \item Вычисление и визуализация частичной суммы ряда Фурье для выбранной волны, с возможностью интерактивного изменения числа учитываемых гармоник. Это демонстрирует приближение исходного сигнала суммой гармоник.
    \item Отображение спектра сигнала (амплитуды гармоник в зависимости от частоты), позволяющее увидеть частотный состав волны.
    \item (\textit{Опционально, если реализовано}) Возможность прослушивания сгенерированного звука непосредственно в браузере с использованием Web Audio API.
    \item (\textit{Опционально, если реализовано}) Возможность загрузки собственного короткого аудиофайла (например, в формате WAV) для анализа его спектрального состава.
\end{itemize}

\textbf{Связь с темой работы.} Веб-приложение 'fourier-wave-analyzer' служит практической реализацией и визуализацией теоретических аспектов, рассмотренных в Главе 1 и Главе 2. Оно позволяет интерактивно исследовать, как изменение параметров (частоты, формы волны, количество гармоник) влияет на вид сигнала и его спектр, делая абстрактные математические концепции рядов Фурье более понятными и осязаемыми. Приложение может быть использовано как образовательный инструмент для студентов, изучающих обработку сигналов, акустику или математический анализ.

Разработка данного веб-инструмента подчеркивает практическую применимость методов Фурье-анализа и демонстрирует возможности современных программных средств для создания интерактивных научных и образовательных ресурсов.

% Если у вас есть скриншот, добавьте его сюда:
% \begin{figure}[ht!]
% \centering
% \includegraphics[width=1\textwidth]{path/to/screenshot.png} % Укажите путь к файлу
% \caption{Интерфейс веб-приложения 'fourier-wave-analyzer'.}
% \label{fig:webapp_screenshot}
% \end{figure}


\conclusion

Выполненная бакалаврская работа достигла поставленной цели --- проведён анализ звуковых волн с использованием рядов Фурье и языка программирования Python. В результате исследования получены следующие основные \textbf{выводы}:
\begin{enumerate}
  \item \textbf{Разложение звуковой волны в ряд Фурье} является эффективным способом описания её частотного содержания. Любая периодическая звуковая волна может быть представлена в виде суммы синусоидальных колебаний (гармоник). В работе на конкретных примерах показано, что сложные по форме сигналы (сумма тонов, прямоугольная волна) разлагаются на гармонические составляющие, совпадающие с теоретически предсказанными.
  \item \textbf{Методы вычислительного анализа на Python} подтвердили свою пригодность. С помощью библиотек NumPy и PyGame был выполнен расчёт дискретных коэффициентов Фурье, генерация и воспроизведение сигналов, что позволило убедиться в адекватности математического описания реальному звуку. Python служит удобной платформой для экспериментальной работы со звуком.
  \item \textbf{Линейность звуковых волн} экспериментально проиллюстрирована через суммирование сигналов, что подтверждает принцип суперпозиции и позволяет моделировать сложные звуки (аккорды) из простых компонентов.
  \item \textbf{Связь временного и частотного описаний} звука подтверждена: анализ в частотной области (спектр) упрощает идентификацию составляющих частот, в то время как временное представление важно для понимания формы волны и переходных процессов.
  \item \textbf{Разработано интерактивное веб-приложение 'fourier-wave-analyzer'}, которое служит практическим инструментом для визуализации и исследования звуковых волн и их разложения в ряд Фурье. Приложение демонстрирует ключевые концепции работы и может быть использовано в образовательных целях.
  \item \textbf{Практические навыки и инструменты} были приобретены: разработаны функции генерации и анализа сигналов, код для визуализации. В приложении представлен структурированный листинг кода.
\end{enumerate}

% (\texttt{fourier\_series}) - Эта строка кажется лишней, возможно, её стоит удалить или проверить её назначение. Я её пока закомментирую.
% % (\texttt{fourier\_series})

\textbf{Научная новизна} работы состоит в том, что она интегрирует классические методы гармонического анализа с современными средствами программирования (Python) и веб-технологиями (разработка 'fourier-wave-analyzer'), демонстрируя их синергию для интерактивного исследования звука. Практическая реализация такого инструмента представляет интерес в контексте популяризации научных методов и образования, способствуя более глубокому усвоению принципов спектрального анализа.

Что касается \textbf{перспектив дальнейших исследований}, они включают:
\begin{itemize}
  \item Расширение анализа на непериодические и реальные сигналы: применение быстрого преобразования Фурье к аудиозаписям, построение спектрограмм.
  \item Исследование влияния фаз гармоник на форму звуковой волны и восприятие.
  \item Автоматизация распознавания нот или тембров в звуковых файлах на основе спектрального анализа.
  \item Синтез новых тембров путем манипуляции коэффициентами ряда Фурье.
  \item Дальнейшее развитие веб-приложения 'fourier-wave-analyzer': добавление новых типов волн, улучшение интерфейса, реализация анализа загружаемых файлов, оптимизация производительности.
\end{itemize}

В заключение, выполненная работа демонстрирует фундаментальное единство математического аппарата Фурье и физической природы звука. Разложение звуковых волн в ряд Фурье позволяет «увидеть» звук, преобразовав его в спектр. Использование Python и разработка веб-приложения сделали этот процесс наглядным и интерактивным, устанавливая живую связь между формулами и звучащей реальностью.

% ... существующий код списка литературы и приложений ...

\begin{thebibliography}{99}

\bibitem{tolstov}
\textbf{Толстов Г. П.} 
Ряды Фурье.  
--- М.: Физматгиз, 1961.  
--- 224 с.

\bibitem{bracewell}
\textbf{Брейсуэлл Р.} 
Преобразование Фурье и его приложения.  
Пер. с англ.  
--- М.: Мир, 1978.  
--- 640 с. 
% (\emph{Оригинал: Bracewell R. The Fourier Transform and Its Applications. 3rd ed. New York: McGraw-Hill, 2000.})

\bibitem{opp-sigsys}
\textbf{Опенгейм А. В., Уиллски А. С., Наваб С. Х.} 
Сигналы и системы.  
2-е изд., пер. с англ.  
--- М.: Техносфера, 2007.  
--- 957 с. 
% (\emph{Оригинал: Oppenheim A. V., Willsky A. S., Nawab S. H. Signals and Systems. 2nd ed. Upper Saddle River, NJ: Prentice Hall, 1997.})

\bibitem{opp-dsp}
\textbf{Опенгейм А. В., Шефер Р. В., Бак Дж. Р.} 
Дискретная обработка сигналов.  
3-е изд., пер. с англ.  
--- М.: Техносфера, 2013.  
--- 1000 с. 
% (\emph{Оригинал: Oppenheim A. V., Schafer R. W., Buck J. R. Discrete-Time Signal Processing. 3rd ed. Upper Saddle River, NJ: Prentice Hall, 2010.})

\bibitem{downey}
\textbf{Дауни А. Б.} 
Think DSP: Цифровая обработка сигналов в Python.  
Пер. с англ.  
--- М.: ДМК Пресс, 2016.  
--- 168 с. 
% (\emph{Оригинал: Downey A. B. Think DSP: Digital Signal Processing in Python. 1st ed. Sebastopol, CA: O'Reilly Media, 2016.})

\bibitem{orland}
\textbf{Орланд П.} 
Математика для программистов: 3D-графика, машинное обучение и моделирование на Python.  
Пер. с англ.  
--- М.: Диалектика, 2022.  
--- 688 с. 
% (\emph{Оригинал: Orland P. Math for Programmers. Shelter Island, NY: Manning, 2021.})

\bibitem{fletcher}
\textbf{Флетчер Н. Х., Россинг Т. Д.} 
Физика музыкальных инструментов.  
2-е изд., пер. с англ.  
--- М.: Мир, 2002.  
--- 756 с. 
% (\emph{Оригинал: Fletcher N. H., Rossing T. D. The Physics of Musical Instruments. 2nd ed. New York: Springer, 1998.})

\bibitem{rossing}
\textbf{Россинг Т. Д., Мур Ф. Р., Уилер П. А.} 
Наука о звуке.  
3-е изд., пер. с англ.  
--- М.: Бином, 2004.  
--- 600 с. 
% (\emph{Оригинал: Rossing T. D., Moore F. R., Wheeler P. A. The Science of Sound. 3rd ed. San Francisco: Addison Wesley, 2002.})

\bibitem{smith-dsp}
\textbf{Смит С. В.} 
Научно-инженерное руководство по цифровой обработке сигналов.  
Пер. с англ.  
--- М.: Солон-Пресс, 1999.  
--- 640 с. 
% (\emph{Оригинал: Smith S. W. The Scientist and Engineer's Guide to Digital Signal Processing. San Diego, CA: California Technical Pub., 1997.})

\bibitem{smith-spectral}
\textbf{Смит III Дж. О.} 
Спектральная обработка аудиосигналов.  
Пер. с англ.  
--- М.: ДМК Пресс, 2012.  
--- 674 с. 
% (\emph{Оригинал: Smith III J. O. Spectral Audio Signal Processing. Palo Alto, CA: W3K Publishing, 2011.})

\bibitem{kotelnikov}
\textbf{Котельников В. А.} 
О пропускной способности эфира и провода в электросвязи.  
--- Москва, 1933.  
% (\emph{Оригинал статьи.})

\bibitem{helmholtz}
\textbf{Гельмгольц Г.} 
Учение о слуховых ощущениях как физиологическая основа для теории музыки.  
Пер. с нем.  
--- Санкт-Петербург: Тип. М. М. Стасюлевича, 1875.  
--- 600 с. 
% (\emph{Оригинал: Helmholtz H. On the Sensations of Tone as a Physiological Basis for the Theory of Music. 2nd English ed. London: Longmans, 1885.})

\bibitem{cooley-tukey}
\textbf{Кули Дж. У., Тьюки Дж. У.} 
Алгоритм машинного вычисления комплексных рядов Фурье // Математика вычислений.  
--- 1965.  
--- 19(90).  
--- С. 297--301. 
% (\emph{Оригинал: Cooley J. W., Tukey J. W. An algorithm for the machine calculation of complex Fourier series. Mathematics of Computation, 1965, 19(90), pp. 297--301.})

\bibitem{bunn}
\textbf{Банн Дж.} 
Георг Симон Ом и математический анализ звука // Isis.  
--- 2008.  
--- 99(1).  
--- C. 50--73. 
% (\emph{Оригинал: Bunn J. Georg Simon Ohm and the mathematical analysis of sound. Isis, 2008, 99(1), pp. 50--73.})

\bibitem{pohlmann}
\textbf{Полман К.} 
Принципы цифрового аудио.  
6-е изд., пер. с англ.  
--- М.: Горячая линия-Телеком, 2011.  
--- 816 с. 
% (\emph{Оригинал: Pohlmann K. C. Principles of Digital Audio. 6th ed. New York: McGraw-Hill, 2010.})

\bibitem{shannon}
\textbf{Шеннон К. Э.} 
Связь в присутствии шума // Труды IRE.  
--- 1949.  
--- 37(1).  
--- С. 10--21. 
% (\emph{Оригинал: Shannon C. E. Communication in the presence of noise. Proc. IRE, 1949, 37(1), pp. 10--21.})

\bibitem{lyons}
\textbf{Лайонс Р. Г.} 
Понимание цифровой обработки сигналов.  
3-е изд., пер. с англ.  
--- М.: Техносфера, 2012.  
--- 933 с. 
% (\emph{Оригинал: Lyons R. G. Understanding Digital Signal Processing. 3rd ed. Upper Saddle River, NJ: Prentice Hall, 2011.})

\bibitem{papoulis}
\textbf{Папулис А.} 
Анализ сигналов.  
Пер. с англ.  
--- М.: Мир, 1979.  
--- 432 с. 
% (\emph{Оригинал: Papoulis A. Signal Analysis. New York: McGraw-Hill, 1977.})

\bibitem{benson}
\textbf{Бенсон Д. Дж.} 
Музыка: математический трактат.  
Пер. с англ.  
--- М.: ЛКИ, 2008.  
--- 426 с. 
% (\emph{Оригинал: Benson D. J. Music: A Mathematical Offering. Cambridge, UK: Cambridge University Press, 2006.})

\bibitem{moore}
\textbf{Мур Б. К. Дж.} 
Введение в психологию слуха.  
5-е изд., пер. с англ.  
--- СПб.: Питер, 2005.  
--- 352 с. 
% (\emph{Оригинал: Moore B. C. J. An Introduction to the Psychology of Hearing. 5th ed. San Diego: Academic Press, 2003.})

\end{thebibliography}

\appendix

\section{Инициализация звуковой системы и генерация шума}
\begin{verbatim}
import pygame, pygame.sndarray
# Инициализация аудио: 44100 Гц, 16-бит, моно
pygame.mixer.init(frequency=44100, size=-16, channels=1)
import numpy as np

# Генерация 1 секунды белого шума 
# (44100 случайных 16-битных значений)
sample_rate = 44100
noise = 
np.random.randint(-32768, 32767, size=sample_rate, dtype=np.int16)
# Воспроизведение шума
noise_sound = pygame.sndarray.make_sound(noise)
noise_sound.play()
\end{verbatim}

\section{Генерация и воспроизведение синусоидальной волны}
\begin{verbatim}
import numpy as np
import math

f = 440  # частота синусоиды, Гц
A = 10000  # амплитуда (в пределах 32767)
N = 44100  # число отсчётов на 1 секунду
n = np.arange(N)
sine_wave = A * np.sin(2 * math.pi * f * n / 44100)
sine_wave = sine_wave.astype(np.int16)

tone = pygame.sndarray.make_sound(sine_wave)
tone.play()
\end{verbatim}

\textit{Комментарии:} Здесь мы генерируем синусоиду 440~Гц и проигрываем её. Параметры можно менять: \texttt{f} — для изменения высоты тона, \texttt{A} — для изменения громкости. После запуска этого кода слышен чистый тон. График такой синусоиды можно построить отдельно (код построения графиков не включён здесь).

\section{Сумма двух синусоид (формирование аккорда)}
\begin{verbatim}
f1 = 440.0   # Hz
f2 = 660.0   # Hz
T = 1.0      # seconds
A1 = 10000
A2 = 10000

N = int(sample_rate * T)
n = np.arange(N)
wave1 = A1 * np.sin(2 * math.pi * f1 * n / sample_rate)
wave2 = A2 * np.sin(2 * math.pi * f2 * n / sample_rate)
wave_sum = wave1 + wave2
wave_sum = wave_sum.astype(np.int16)

chord = pygame.sndarray.make_sound(wave_sum)
chord.play()
\end{verbatim}

\section{Код для вычисления коэффициентов Фурье}

\begin{verbatim}
from math import sin, cos, pi, sqrt
import numpy as np

def const(t):
    return 1 / sqrt(2)

def s(n):
    def f(t):
        return sin(2*pi*n*t)
    return f

def c(n):
    def f(t):
        return cos(2*pi*n*t)
    return f

def inner_product(f, g, N=1000):
    dt = 1/N
    return 2*sum([f(t)*g(t)*dt
                  for t in np.arange(0,1,dt)])

def fourier_coefficients(f, N):
    a0 = inner_product(f, const)
    an = [inner_product(f, c(n))
          for n in range(1, N+1)]
    bn = [inner_product(f, s(n))
          for n in range(1, N+1)]
    return a0, an, bn
\end{verbatim}

Данный код реализует вычисление коэффициентов Фурье для периодической функции с периодом 1. Функция \texttt{inner\_product} вычисляет внутреннее произведение двух функций методом численного интегрирования. Функции \texttt{s} и \texttt{c} создают синусоидальные и косинусоидальные функции с заданной частотой. Функция \texttt{const} определяет постоянную базисную функцию со значением $1/\sqrt{2}$.

Функция \texttt{fourier\_coefficients} принимает на вход функцию $f$ и число $N$, обозначающее максимальное количество гармоник, и возвращает коэффициенты ряда Фурье: постоянный член $a_0$ и два списка коэффициентов $a_n$ и $b_n$. Эти коэффициенты представляют компоненты функции $f$ в базисе, состоящем из постоянной функции и функций синуса и косинуса различных частот.

\section{Вычисление дискретного спектра (FFT) и анализ коэффициентов}

\begin{verbatim}
import numpy.fft as fft
X = fft.fft(wave_sum)  # комплексный спектр
amplitudes = np.abs(X) / (N/2.0)
freqs = np.linspace(0, sample_rate/2, N//2, endpoint=False)
amp_half = amplitudes[:N//2]  # спектр для положительных частот
peaks = []
for i in range(1, len(amp_half)-1):
    if amp_half[i] > amp_half[i-1] and amp_half[i] > amp_half[i+1]:
        if amp_half[i] > 100:  # порог, чтобы отсечь шумовые пики
            peaks.append((freqs[i], amp_half[i]))
print("Пиковые частоты и амплитуды:", peaks)
\end{verbatim}
\textit{Комментарии:} Здесь производится быстрое преобразование Фурье (\texttt{fft.fft}) для сигнала \texttt{wave\_sum} (суммы двух тонов). Затем вычисляются амплитуды (модуль комплексных коэффициентов, с нормировкой, чтобы амплитуды синусоид соответствовали оригинальным). После этого простой алгоритм находит пики в первой половине спектра (положительные частоты). Вывод (\texttt{print}) покажет результат, близкий к \texttt{[(440.0, A1), (660.0, A2)]} (с учётом дискретизации возможна погрешность в несколько Гц). Этот код демонстрирует, как автоматически извлечь частоты из сигнала. Аналогично можно анализировать другие сигналы (например, \texttt{sine} — будет один пик на 440~Гц, \texttt{noise} — не будет выраженных пиков и т.д.).

\end{document}
